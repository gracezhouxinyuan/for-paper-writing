\documentclass[withoutpreface,bwprint]{cumcmthesis}

\usepackage[framemethod=TikZ]{mdframed}
\usepackage{url}   % 网页链接
\usepackage{subcaption} % 子标题
\usepackage{booktabs}
\usepackage{makecell}%用于控制线宽的宏包
\newcolumntype{L}{>{\raggedright\arraybackslash}X} 
\title{全国大学生数学建模竞赛模板}
\tihao{A}
\baominghao{4321}
\schoolname{XX大学}
\membera{ }
\memberb{ }
\memberc{ }
\supervisor{ }%辅导老师
\yearinput{2023}
\monthinput{9}
\dayinput{8}

\begin{document}
\maketitle
\begin{abstract}
\textbf{对于问题一,}

\textbf{对于问题二,}

\textbf{对于问题三,}

\textbf{对于问题四,}

\begin{enumerate}
    \item 一句话陈述了本文研究的一般问题
    \item 一两句话总结了整体方法、主要结果和结论
    \item 一两句话回到一般背景,解释比以前的工作取得的进展,并概述针对各个问题展开介绍“方法+结果+优势”,例如我们可以说说自己避免了哪些问题,误差值是多少云云
\end{enumerate}

\keywords{\quad \quad  \quad  }

\end{abstract}

\tableofcontents

\newpage

\section{问题重述}

\subsection{问题背景}

\subsection{问题提出}

\textbf{问题1:}  

\textbf{问题2:}  

\textbf{问题3:} 

\textbf{问题4:}  

%%%%%%%%%%%%%%%%%%%%%%%%%%%%%%%%%%%%%%%%%%%%%%%%%%%%%%%%%%%%%

\section{问题分析}

\subsection{问题一分析}
对于问题一,

\subsection{问题二分析}	
对于问题二,

\subsection{问题三分析}
对于问题三,

\subsection{问题四分析}
对于问题四,

%%%%%%%%%%%%%%%%%%%%%%%%%%%%%%%%%%%%%%%%%%%%%%%%%%%%%%%%%%%%% 

\section{模型假设}
\begin{itemize}[itemindent=2em]
\item 假设1
\item 假设2
\item 假设3
\item 假设4
\end{itemize}

%%%%%%%%%%%%%%%%%%%%%%%%%%%%%%%%%%%%%%%%%%%%%%%%%%%%%%%%%%%%% 

\section{符号说明}
\begin{tabularx}{\textwidth}{LL}
    \toprule
    符号&含义\\ \midrule
    $s_i$&第i种投资项目\\
    $r_i$&$s_i$的平均收益率\\
    $p_i$&$s_i$的交易费率\\
    $q_i$&$s_i$的风险损失率\\
    $u_i$&$s_i$的交易定额\\
    $r_0$&同期银行利率\\
    $x_i$&投资项目$s_i$的资金\\
    a&总体风险度\\
    Q&总体收益\\ \bottomrule
\end{tabularx}

\section{问题一的模型的建立和求解}

\subsection{模型建立}

\subsubsection{模型1名称}

\subsubsection{模型2名称}

\subsection{模型求解结果}

不光呈现结果,要进行仿真绘图和数据绘图,和前文的假设或者直观感受相符。

还可以进行灵敏度的分析。
% %%%%%%%%%%%%%%%%%%%%%%%%%%%%%%%%%%%%%%%%%%%%%%%%%%%%%%%%%%%%% 

\section{问题二的模型的建立和求解}

\subsection{模型建立}


\subsubsection{模型1名称}



\subsubsection{模型2名称}



\subsection{模型求解结果}

问题二结果及分析。

%%%%%%%%%%%%%%%%%%%%%%%%%%%%%%%%%%%%%%%%%%%%%%%%%%%%%%%%%%%%% 

\section{问题三的模型的建立和求解}

\subsection{模型建立}

\subsubsection{模型1名称}

\subsubsection{模型2名称}

\subsection{模型求解结果}

问题三结果及分析。

%%%%%%%%%%%%%%%%%%%%%%%%%%%%%%%%%%%%%%%%%%%%%%%%%%%%%%%%%%%%% 

\section{问题四的模型的建立和求解}

\subsection{模型建立}

\subsubsection{模型1名称}

\subsubsection{模型2名称}

\subsection{模型求解结果}

问题四结果及分析。

%%%%%%%%%%%%%%%%%%%%%%%%%%%%%%%%%%%%%%%%%%%%%%%%%%%%%%%%%%%%%

\section{模型的评价}


\subsection{模型的优点}
\begin{itemize}[itemindent=2em]
\item 优点1
\item 优点2
\item 优点3
\end{itemize}

\subsection{模型的缺点}
\begin{itemize}[itemindent=2em]
\item 缺点1
\item 缺点2
\end{itemize}


%%%%%%%%%%%%%%%%%%%%%%%%%%%%%%%%%%%%%%%%%%%%%%%%%%%%%%%%%%%%%
%% 参考文献


\nocite{*}

\begin{thebibliography}{99} 

% 遵循 GB/T 7714-2015
\bibitem{ref1} 昂温 G, 斯特朗 G. 概率论与数理统计[M]. 2版. 北京: 科学出版社, 2002.
\bibitem{ref2} 张三, 李四. 关于A算法的优化研究[J]. 计算机学报, 2022, 45(10): 2100-2110.
\end{thebibliography}




\newpage
%%%%%%%%%%%%%%%%%%%%%%%%%%%%%%%%%%%%%%%%%%%%%%%%%%%%%%%%%%%%%
%% 附录
\begin{appendices}
\section{文件列表}
\begin{table}[H]
\centering
\begin{tabularx}{\textwidth}{LL}
\toprule
文件名   & 功能描述 \\
\midrule
P1.py & 问题一程序代码 \\
P2.py & 问题二程序代码 \\
P3.py & 问题三程序代码 \\
P4.py & 问题四程序代码 \\
\bottomrule
\end{tabularx}
\label{tab:文件列表}
\end{table}

\section{代码}
\noindent P1.py

% 请把代码文件放在code文件夹里并写清楚路径
%\lstinputlisting[language=python]{code/q2.py}
%P2.py
%\lstinputlisting[language=python]{code/q2.py}
%P3.py

\end{appendices}

\end{document} 
