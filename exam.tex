\documentclass[withoutpreface,bwprint]{cumcmthesis}

\usepackage[framemethod=TikZ]{mdframed}
\usepackage{url}   % 网页链接
\usepackage{subcaption} % 子标题
\usepackage{booktabs}
\usepackage{makecell}%用于控制线宽的宏包
\usepackage{algorithm}
\usepackage{algpseudocode}
\usepackage{amsmath}
\usepackage{xcolor}%用于创建伪代码的宏包
\newcolumntype{L}{>{\raggedright\arraybackslash}X} 
\title{全国大学生数学建模竞赛模板}
\tihao{A}
\baominghao{4321}
\schoolname{XX大学}
\membera{ }
\memberb{ }
\memberc{ }
\supervisor{ }%辅导老师
\yearinput{2023}
\monthinput{9}
\dayinput{8}

\begin{document}
\maketitle
\begin{abstract}
    本文通过研究。。。,提出了。。。

\textbf{对于问题一,}
该方案首先基于。。。,设计了一个模型,给出了直接计算。。。的公式。

\textbf{对于问题二,}

\textbf{对于问题三,}

\textbf{对于问题四,}

\begin{enumerate}
    \item 一句话陈述了本文研究的一般问题
    \item 一两句话总结了整体方法、主要结果和结论
    \item 一两句话回到一般背景,解释比以前的工作取得的进展,并概述针对各个问题展开介绍“方法+结果+优势”,例如我们可以说说自己避免了哪些问题,误差值是多少云云
\end{enumerate}

\keywords{\quad \quad  \quad  }

\end{abstract}

\tableofcontents

\newpage

\section{问题的背景与重述}

\subsection{问题的背景}

\subsection{问题的提出}

\textbf{问题1:}  

\textbf{问题2:}  

\textbf{问题3:} 

\textbf{问题4:}  

%%%%%%%%%%%%%%%%%%%%%%%%%%%%%%%%%%%%%%%%%%%%%%%%%%%%%%%%%%%%%

\section{问题分析}

\subsection{问题一}
对于问题一,

\subsection{问题二}	
对于问题二,

\subsection{问题三}
对于问题三,

\subsection{问题四}
对于问题四,

%%%%%%%%%%%%%%%%%%%%%%%%%%%%%%%%%%%%%%%%%%%%%%%%%%%%%%%%%%%%% 

\section{模型假设}
模型假设主要包括

\begin{itemize}[itemindent=2em]
\item 假设1
\item 假设2
\item 假设3
\item 假设4
\end{itemize}

%%%%%%%%%%%%%%%%%%%%%%%%%%%%%%%%%%%%%%%%%%%%%%%%%%%%%%%%%%%%% 

\section{符号说明}
\begin{tabularx}{\textwidth}{LL}
    \toprule
    符号&含义\\ \midrule
    $s_i$&第i种投资项目\\
    $r_i$&$s_i$的平均收益率\\
    $p_i$&$s_i$的交易费率\\
    $q_i$&$s_i$的风险损失率\\
    $u_i$&$s_i$的交易定额\\
    $r_0$&同期银行利率\\
    $x_i$&投资项目$s_i$的资金\\
    a&总体风险度\\
    Q&总体收益\\ \bottomrule
\end{tabularx}

%%%%%%%%%%%%%%%%%%%%%%%%%%%%%%%%%%%%%%%%%%%%%%%%%%%%%%%%%%%%% 

\section{模型的建立和求解}

\subsection{问题1的模型建立}

\subsubsection{模型1名称}

\begin{algorithm}[H]
\caption{Optimal Order Quantity Simulation for Newsvendor Problem}
\label{alg:newsvendor}
\begin{algorithmic}[1]
\Require Unit price $a$, unit cost $b$, Poisson parameter $\lambda$, number of simulations $N$
\Ensure Optimal order quantity $u^*$
\Statex
\State \textbf{Initialization:}
\State $M_{\text{prev}} \gets 0$ \Comment{Expected profit for order quantity $u=0$}
\State $u \gets 1$ \Comment{Start testing from order quantity $u=1$}
\Statex

\While{True}
    \State $\text{total\_profit} \gets 0$
    \Statex
    \For{$i \gets 1$ \textbf{to} $N$} \Comment{Simulation phase}
        \State Generate random demand $X \sim \text{Poisson}(\lambda)$
        \State $\text{sales} \gets \min(X, u)$
        \State $\text{profit} \gets a \times \text{sales} - b \times u$
        \State $\text{total\_profit} \gets \text{total\_profit} + \text{profit}$
    \EndFor
    \Statex
    \State $M_{\text{current}} \gets \text{total\_profit} / N$ \Comment{Calculate average expected profit}
    \Statex
    \If{$M_{\text{current}} < M_{\text{prev}}$}
        \State $u^* \gets u - 1$ \Comment{Optimal quantity found}
        \State \textbf{break} \Comment{Exit the loop}
    \Else
        \State $M_{\text{prev}} \gets M_{\text{current}}$ \Comment{Update previous profit}
        \State $u \gets u + 1$ \Comment{Test next order quantity}
    \EndIf
\EndWhile
\Statex
\State \textbf{return} $u^*$
\end{algorithmic}
\end{algorithm}

\subsubsection{模型2名称}

\subsubsection{模型求解方法名称}

\subsubsection{模型总结}

%%%%%%%%%%%%%%%%%%%%%%%%%%%%%%%%%%%%%%%%%%%%%%%%%%%%%%%%%%%%% 

\section{模型评价与展望}


\subsection{模型优点}
\begin{itemize}[itemindent=2em]
\item 优点1
\item 优点2
\item 优点3
\end{itemize}

\subsection{模型缺点}
\begin{itemize}[itemindent=2em]
\item 缺点1
\item 缺点2
\end{itemize}

\subsection{未来展望}
\begin{itemize}[itemindent=2em]
\item 展望1
\item 展望2
\end{itemize}

%%%%%%%%%%%%%%%%%%%%%%%%%%%%%%%%%%%%%%%%%%%%%%%%%%%%%%%%%%%%%
%% 参考文献


\nocite{*}

\begin{thebibliography}{99} 

% 遵循 GB/T 7714-2015
\bibitem{ref1} 昂温 G, 斯特朗 G. 概率论与数理统计[M]. 2版. 北京: 科学出版社, 2002.
\bibitem{ref2} 张三, 李四. 关于A算法的优化研究[J]. 计算机学报, 2022, 45(10): 2100-2110.
\bibitem{ref1} 作者. 文献标题[J]. 期刊名, 年份, 卷(期): 页码.
\bibitem{ref2} 作者. 书名[M]. 出版地: 出版社, 年份.
\end{thebibliography}




\newpage
%%%%%%%%%%%%%%%%%%%%%%%%%%%%%%%%%%%%%%%%%%%%%%%%%%%%%%%%%%%%%
%% 附录
\begin{appendices}
\section{文件列表}
\begin{table}[H]
\centering
\begin{tabularx}{\textwidth}{LL}
\toprule
文件名   & 功能描述 \\
\midrule
P1.py & 问题一程序代码 \\
P2.py & 问题二程序代码 \\
P3.py & 问题三程序代码 \\
P4.py & 问题四程序代码 \\
\bottomrule
\end{tabularx}
\label{tab:文件列表}
\end{table}

\section{代码}
\noindent P1.py

% 请把代码文件放在code文件夹里并写清楚路径
%\lstinputlisting[language=python]{code/q2.py}
%P2.py
%\lstinputlisting[language=python]{code/q2.py}
%P3.py

\end{appendices}

\end{document} 