\documentclass[UTF8]{ctexart}
\usepackage{graphicx}%用于插入图片的宏包
\title{杂谈勾股定理}
\author{张三}
\date{\today}

\bibliographystyle{plain}

\begin{document}

\maketitle%创建标题
\tableofcontents%创建标题
\begin{abstract}%摘要部分
    这是一篇关于勾股定理的小短文
\end{abstract}
\section{勾股定理在古代}%创建模块
\begin{equation}
    a(b+c)=ab+ac
\end{equation}

\begin{equation}
    AB^2=BC^2+AC^2
\end{equation}

%\includegraphics[width=3cm]{02勾股_xiantu.png}
%用于插入图片的指令
%width宽度,scale放缩因子,height高度

\begin{figure}[ht]
    \centering
    \includegraphics[scale=0.6]{02勾股_xiantu.png}
    \caption{宋赵爽在周髀算经注中作的弦图}
    \label{fig:xiantu}
\end{figure}
%通过浮动体插入图片(便于控制位置,便于插入说明性的标题)

\section{勾股定理在近代的形式}
\emph{勾股数}%强调内容

\begin{quote}%引用内容
    \zihao{-5}\kaishu 引用的内容%改变引用部分的字体
    勾广三,股修四,径隅五
\end{quote}
\end{document}