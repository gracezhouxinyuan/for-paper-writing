\documentclass[UTF8]{ctexart}
\usepackage{booktabs}
\begin{document}

\begin{tabular}{lcr}
%第二个花括号中的字母表示对齐格式(left,center,right)
    left&center&right\\
    本列左对齐&本列局中&本列右对齐\\
%  每行后用\\换行,一行之内的不同列用&分开 
\end{tabular}

\[
    \begin{array}{c|c}
        %表格中可选的垂直对齐参数为:t:表格顶部对齐 b:表格底部对齐
        \frac12&0\\
        \hline
        0&-\frac12\\        
    \end{array}
\]

\begin{tabular}{|rr|}
\hline
    输入&输出\\ \hline
    $-2$ &4\\
    0&0\\
    2&4\\
    \hline
\end{tabular}
\qquad
输入与输出有关系$y=x^2$

\begin{tabular}{|c|rrr|p{4em}|}
\hline
    姓名&语文&数学&外语&备注\\
\hline
    张三&87&100&93&优秀\\
    李四&75&63&52&补考另行通知\\
    王小二&80&82&78&\\
\hline
\end{tabular}

%含小数点的表格
\begin{tabular}{|c|r@{.}l|}%可以实现小数点对齐!
    \hline
    收入& 12345&6 \\ \hline
    支出& 765&43 \\ \hline
    节余& 11580&17 \\ \hline
\end{tabular}


\end{document}



